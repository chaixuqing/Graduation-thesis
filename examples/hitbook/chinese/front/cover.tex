% !Mode:: "TeX:UTF-8"

\hitsetup{
  %******************************
  % 注意:
  %   1. 配置里面不要出现空行
  %   2. 不需要的配置信息可以删除
  %******************************
  %
  %=====
  % 秘级
  %=====
  statesecrets={公开},
  natclassifiedindex={TM301.2},
  intclassifiedindex={62-5},
  %
  %=========
  % 中文信息
  %=========
  ctitleone={面向动态业务的虚拟},%本科生封面使用
  ctitletwo={网络功能转发图扩展算法},%本科生封面使用
  ctitlecover={NFV:下一代移动网络(VEPC)的现状、挑战和实施},%放在封面中使用,自由断行
  ctitle={NFV:下一代移动网络(VEPC)的现状、挑战和实施},%放在原创性声明中使用
  csubtitle={一条副标题}, %一般情况没有,可以注释掉
  cxueke={工学},
  csubject={信息安全},
  caffil={计算机科学与技术学院},
  cauthor={柴绪清},
  csupervisor={李宁},
  cassosupervisor={某某某教授}, % 副指导老师
  ccosupervisor={某某某教授}, % 联合指导老师
  % 日期自动使用当前时间,若需指定按如下方式修改:
  cdate={2021.06.16},
  cstudentid={171210202},
  % cstudenttype={同等学力人员}, %非全日制教育申请学位者
  % cnumber={no9527}, %编号
  % cpositionname={哈铁西站}, %博士后站名称
  % cstartdate={3050年9月10日}, %到站日期
  % cenddate={3090年10月10日}, %出站日期
  %(同等学力人员)、(工程硕士)、(工商管理硕士)、
  %(高级管理人员工商管理硕士)、(公共管理硕士)、(中职教师)、(高校教师)等
  %
  %
  %=========
  % 英文信息
  %=========
  etitle={Research on key technologies of partial porous externally pressurized gas bearing},
  esubtitle={This is the sub title},
  exueke={Engineering},
  esubject={Computer Science and Technology},
  eaffil={\emultiline[t]{School of Mechatronics Engineering \\ Mechatronics Engineering}},
  eauthor={Yu Dongmei},
  esupervisor={Professor XXX},
  eassosupervisor={XXX},
  % 日期自动生成,若需指定按如下方式修改:
  edate={December, 2017},
  estudenttype={Master of Art},
  %
  % 关键词用“英文逗号”分割
  ckeywords={网络功能虚拟化,虚拟网络功能转发图, 特征分解,启发式算法},
  ekeywords={NFV, VNF-FG, Eigendecomposition,Heuristic algorithm},
}

\begin{cabstract}
  随着移动网络用户期待5G网络的连接速度,服务提供商面临着在没有大量财务投资的情况下满足连接需求的挑战。网络功能虚拟化(NFV)作为一种新的方法被引入,它提供了一种摆脱这一瓶颈的方法。NFV准备改变电信基础设施的核心结构,使其更具成本效益。本文介绍了一种NFV框架,并讨论了其在移动网络中应用的挑战和要求。特别提出了一种虚拟环境下的NFV框架。此外,为了减少信令流量并获得更好的性能,本文提出了一种将虚拟化演进分组核心的多种功能捆绑在单个物理设备或一组相邻设备上的标准。分析表明,该分组可以减少70\%的网络控制业务量。
\end{cabstract}

\begin{eabstract}
  Traditional network architectures need to deploy dedicated physical equipment (firewalls, encryption equipment, etc.) when providing network services. The network function and the physical device are tightly coupled. Each network function is implemented by a specific hardware platform, which has poor flexibility and scalability. The upgrade and maintenance of network functions are all restricted by the equipment provider. The network function virtualization technology attempts to solve the above-mentioned problems by reconstructing the existing network architecture. Network function virtualization decouples the software implementation of network functions from physical hardware. The virtual network function can be linked with other VNFs and physical network functions to realize network services. Tenants can deploy various services and network functions in the physical infrastructure, while service providers have dynamic VNF-FG establishment, update, and expansion requirements. Therefore, infrastructure providers not only need to delete, replace, insert, and add VNF instances, but also need to extend the service function chain that has been instantiated without interfering with the previously deployed service function chain. The expansion of VNF-FG must be realized without interrupting the initially deployed service instance, and must be seamlessly connected to applications and services.\par
  This paper first proposes a heuristic algorithm based on eigendecomposition to solve the problem of virtualized network function forwarding graph expansion. This algorithm provides an expansion scheme close to the optimal solution while reducing the complexity of the algorithm. After that, the linear programming solver is used to solve the virtual network function forwarding graph expansion problem, and this algorithm is used as a reference for comparing the final results of heuristic algorithms. Finally, the calculation results of the two algorithms are compared in the expansion success rate and objective function.
\end{eabstract}
