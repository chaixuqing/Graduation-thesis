\chapter[绪 论]{绪 论}[]
降低资本支出(CAPEX)和运营支出(OPEX)的需求促使信息技术(IT)专业人士更多地考虑如何实现更有效的资本投资和更高的资本回报。为了实现这一目标,虚拟化技术已经成为一种将软件应用程序与底层硬件解耦并使软件能够在虚拟化环境中运行的方法。在虚拟环境中,硬件被仿真,操作系统(OS)在仿真的硬件上运行,就像它在自己的裸机资源上运行一样。在此过程中,多个虚拟机可以共享可用资源并同时在一台物理机上运行[1]。在过去的十年中,对宽带网络连接的需求急剧增加。随着连接互联网的移动设备(从智能手机、平板电脑和笔记本电脑到传感器网络和机器到机器(M2M)连接)数量的增加,它获得了额外的发展势头。这种日益增长的需求正在推动网络服务提供商投资基础设施以跟上需求,尽管研究表明这种投资的回报微乎其微[2]。网络支出在很大程度上取决于网络所依赖的基础设施。任何网络改进、升级或发布新服务的高昂成本都会拉高服务提供商的利润空间。网络运营挑战不仅限于昂贵的硬件设备成本,还包括不断增加的能源成本和对高素质人才的竞争市场,以及设计、集成和运营日益复杂的基于硬件的基础设施所需的技能。此外,管理网络基础设施是服务提供商的另一个主要关注点。这些问题不仅会影响收入,还会增加上市时间,限制电信行业的创新。为了成功实现这些目标,由七家电信运营商组成的小组在欧洲电信标准协会(ETSI)下成立了网络功能虚拟化(NFV)行业规范小组。他们在2012年10月公布了他们的解决方案[3]。最近,几家电信设备提供商和IT专家加入了这个小组,本文介绍了NFV,并提供了在电信网核心网中设计和实施NFV的指导方针。我们给出了NFV的概念定义。我们将讨论NFV框架。提供了一种实现NFV框架实体的方法。讨论了NFV面临的挑战和要求。我们定义了NFV在移动网络中的不同服务和用例。讨论了虚拟演进分组核心(VEPC)网络。介绍了NFV环境下EPC实体的分组方法